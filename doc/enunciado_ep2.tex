%% This is file `elsarticle-template-1-num.tex',
%%
%% Copyright 2009 Elsevier Ltd
%%
%% This file is part of the 'Elsarticle Bundle'.
%% ---------------------------------------------
%%
%% It may be distributed under the conditions of the LaTeX Project Public
%% License, either version 1.2 of this license or (at your option) any
%% later version.  The latest version of this license is in
%%    http://www.latex-project.org/lppl.txt
%% and version 1.2 or later is part of all distributions of LaTeX
%% version 1999/12/01 or later.
%%
%% Template article for Elsevier's document class `elsarticle'
%% with numbered style bibliographic references
%%
%% $Id: elsarticle-template-1-num.tex 149 2009-10-08 05:01:15Z rishi $
%% $URL: http://lenova.river-valley.com/svn/elsbst/trunk/elsarticle-template-1-num.tex $
%%
\documentclass[final,12pt,a4paper]{elsarticle}

%% Use the option review to obtain double line spacing
%% \documentclass[preprint,review,12pt]{elsarticle}

%% Use the options 1p,twocolumn; 3p; 3p,twocolumn; 5p; or 5p,twocolumn
%% for a journal layout:
%% \documentclass[final,1p,times]{elsarticle}
%% \documentclass[final,1p,times,twocolumn]{elsarticle}
%% \documentclass[final,3p,times]{elsarticle}
%% \documentclass[final,3p,times,twocolumn]{elsarticle}
%% \documentclass[final,5p,times]{elsarticle}
%% \documentclass[final,5p,times,twocolumn]{elsarticle}

%% The graphicx package provides the includegraphics command.
\usepackage{graphicx}
%% The amssymb package provides various useful mathematical symbols
\usepackage{amssymb}
%% The amsthm package provides extended theorem environments
%% \usepackage{amsthm}

\usepackage{booktabs}
\usepackage{xcolor}
\usepackage{sourcecodepro}
\usepackage{url}
\usepackage{listings}
\usepackage[utf8]{inputenc}
\usepackage[brazilian]{babel}
\usepackage{multirow}
\usepackage{textcomp}
\usepackage{caption}

\definecolor{Accent}{HTML}{157FFF}

\lstdefinestyle{customMtheme}{%
  backgroundcolor={},
  basicstyle=\ttfamily\scriptsize,
  breakatwhitespace=true,
  breaklines=true,
  captionpos=n,
  commentstyle=\color{orange},
  escapeinside={\%*}{*)},
  extendedchars=true,
  frame=n,
  keywordstyle=\color{Accent},
  language=C++,
  rulecolor=\color{black},
  showspaces=false,
  showstringspaces=false,
  xleftmargin=.5cm,
  xrightmargin=.5cm,
  showtabs=false,
  stepnumber=2,
  stringstyle=\color{gray},
  tabsize=4,
  keywords={void, int, float, main,
  if, else, malloc, NULL,
  fprintf, stderr, for, make, gcc, o, Enter, Ctrl},
  otherkeywords={\#pragma, \#include, \&, \*, +, -, /, [, ], >, <, \$, \., std\=c11}
}
\lstset{basicstyle=\ttfamily\scriptsize,style=customMtheme}

\renewcommand*{\UrlFont}{\ttfamily\scriptsize\relax}

\graphicspath{{./img/}}

%% The lineno packages adds line numbers. Start line numbering with
%% \begin{linenumbers}, end it with \end{linenumbers}. Or switch it on
%% for the whole article with \linenumbers after \end{frontmatter}.
%% \usepackage{lineno}

%% natbib.sty is loaded by default. However, natbib options can be
%% provided with \biboptions{...} command. Following options are
%% valid:

%%   round  -  round parentheses are used (default)
%%   square -  square brackets are used   [option]
%%   curly  -  curly braces are used      {option}
%%   angle  -  angle brackets are used    <option>
%%   semicolon  -  multiple citations separated by semi-colon
%%   colon  - same as semicolon, an earlier confusion
%%   comma  -  separated by comma
%%   numbers-  selects numerical citations
%%   super  -  numerical citations as superscripts
%%   sort   -  sorts multiple citations according to order in ref. list
%%   sort&compress   -  like sort, but also compresses numerical citations
%%   compress - compresses without sorting
%%
%% \biboptions{comma,round}

% \biboptions{}

%% Removing lines when no abstract is given
\makeatletter
\renewcommand{\MaketitleBox}{%
    \resetTitleCounters
        \def\baselinestretch{1}%
        \begin{center}
    \def\baselinestretch{1}%
        \Large \@title \par
        \vskip 18pt
        \normalsize\elsauthors \par
        \vskip 10pt
        \footnotesize \itshape \elsaddress \par
        \end{center}
    \vskip 12pt
}
\makeatother

%% Removing custom footer on fist page
\makeatletter
\def\ps@pprintTitle{%
    \let\@oddhead\@empty
        \let\@evenhead\@empty
        \def\@oddfoot{\centerline{\thepage}%
        }%
    \let\@evenfoot\@oddfoot
}%
\makeatother

\journal{MAC 5742-0219 Introdução à Programação Concorrente, Paralela e Distribuída}

\begin{document}

\begin{frontmatter}

%% Title, authors and addresses

\title{EP2: Criptografia em GPUs usando CUDA}

%% use the tnoteref command within \title for footnotes;
%% use the tnotetext command for the associated footnote;
%% use the fnref command within \author or \address for footnotes;
%% use the fntext command for the associated footnote;
%% use the corref command within \author for corresponding author footnotes;
%% use the cortext command for the associated footnote;
%% use the ead command for the email address,
%% and the form \ead[url] for the home page:
%%
%% \title{Title\tnoteref{label1}}
%% \tnotetext[label1]{}
%% \author{Name\corref{cor1}\fnref{label2}}
%% \ead{email address}
%% \ead[url]{home page}
%% \fntext[label2]{}
%% \cortext[cor1]{}
%% \address{Address\fnref{label3}}
%% \fntext[label3]{}


%% use optional labels to link authors explicitly to addresses:
%% \author[label1,label2]{<author name>}
%% \address[label1]{<address>}
%% \address[label2]{<address>}

\author{Pedro Bruel e Alfredo Goldman}

\address{MAC 5742-0219 Introdução à Programação Concorrente, Paralela e Distribuída}

%%\begin{abstract}
%% Text of abstract
%% Suspendisse potenti. Suspendisse quis sem elit, et mattis nisl. Phasellus
%% consequat erat eu velit rhoncus non pharetra neque auctor. Phasellus eu lacus
%% quam. Ut ipsum dolor, euismod aliquam congue sed, lobortis et orci. Mauris eget
%% velit id arcu ultricies auctor in eget dolor. Pellentesque suscipit adipiscing
%% sem, imperdiet laoreet dolor elementum ut. Mauris condimentum est sed velit
%% lacinia placerat. Vestibulum ante ipsum primis in faucibus orci luctus et
%% ultrices posuere cubilia Curae; Nullam diam metus, pharetra vitae euismod sed,
%% placerat ultrices eros. Aliquam tincidunt dapibus venenatis. In interdum tellus
%% nec justo accumsan aliquam. Nulla sit amet massa augue.
%% \end{abstract}
%%
%% \begin{keyword}
%% Science \sep Publication \sep Complicated
%% keywords here, in the form: keyword \sep keyword

%% MSC codes here, in the form: \MSC code \sep code
%% or \MSC[2008] code \sep code (2000 is the default)

%% \end{keyword}

\end{frontmatter}

%%
%% Start line numbering here if you want
%%
%% \linenumbers

%% main text
\section{Introdução}

Neste EP vocês implementarão versões para GPU, usando CUDA, de alguns
algoritmos simples de encriptação e geração de \textit{hashes} de arquivos.
Forneceremos implementações em \texttt{C} dos algoritmos, e cada grupo
deverá implementar no mínimo 3 algoritmos usando CUDA.

O código na linguagem \texttt{C} e os arquivos em \LaTeX{} necessários para
gerar este documento estão disponíveis no
\textit{GitHub}\footnote{\url{https://github.com/phrb/MAC5742-0219-EP2}
[Acessado em 11/05/2017]}. O resto deste documento descreve as tarefas que você
e seu grupo deverão realizar no EP2.

\section{Tarefas}

Vocês devem implementar em CUDA no mínimo 3 dos algoritmos de encriptação ou
\textit{hash} disponibilizados no repositório do EP.
Vocês podem usar 3 algoritmos de encriptação ou \textit{hash} de outras fontes,
\textbf{desde que entreguem também o código sequencial do algoritmo}.

Depois, devem realizar experimentos comparando o desempenho das versões
sequenciais e em CUDA, usando os arquivos de teste disponibilizados no
repositório, ou arquivos escolhidos por vocês. Caso usem arquivos diferentes
dos disponibilizados, incluam \textit{links} no relatório para lugares onde os
arquivos estão hospedados.

O arquivo \texttt{src/crypto-algorithms/des\_test.c} contém uma função
para ler e escrever um arquivo como um \textit{stream} de bytes. Vocês
podem usar essa função para encriptar arquivos ou escrever a sua própria.
Vocês devem escrever \textbf{funções de teste} que comparem os arquivos antes
da encriptação com os arquivos após uma encriptação e decriptação,
ou com o \textit{hash} do arquivo.

Vocês devem entregar um relatório contento gráficos e análises de desempenho
dos seus programas e das versões sequenciais. Vocês devem justificar, usando
dados e gráficos, as escolhas para tamanho de \textit{block} e \textit{grid}
dos seus programas em CUDA.

\subsection{Apresentação dos Resultados}

Depois de realizar os experimentos vocês deverão elaborar gráficos que
evidenciem o comportamento dos três algoritmos com relação à variação do
tamanho do arquivo processado. Os gráficos deverão ser claros e legíveis, com
eixos nomeados. Deverão apresentar a média e o desvio padrão de $10$ execuções
de cada programa em cada arquivo.

Recomendamos que vocês usem ferramentas como a biblioteca \texttt{matplotlib}
da linguagem \texttt{Python}. Se fizerem isso vocês conseguirão automatizar a
realização dos experimentos e a geração dos gráficos. A automatização dos
experimentos e da visualização dos dados gerados é fundamental para pesquisa em
Ciência da Computação, pois permite gerar e analisar grandes conjuntos de dados
sem muito esforço manual.

\subsection{Discussão dos Resultados}

Vocês deverão analisar os resultados obtidos e tentar responder a algumas
perguntas, para cada um dos 3 algoritmos:

\begin{itemize}
    \item Como o tempo de execução varia conforme o tamanho do arquivo?
    \item Como e por quais razões vocês escolheram o tamanho de \textit{block}
        e \textit{grid}?
    \item Qual o impacto das operações de I/O e alocação de memória no tempo de
    execução?
\end{itemize}

Vocês conseguem pensar em mais perguntas interessantes?

\subsection{Entrega no PACA}

Vocês deverão entregar no PACA \textbf{apenas um relatório e código fonte por
grupo}. A entrega deve ser um único arquivo nos formatos \texttt{.tar},
\texttt{.zip}, ou qualquer formato que o \texttt{tar} consiga descompactar.
A entrega deve ser feita \textbf{até dia 01/06/17}, e deve conter o código
em CUDA, código sequencial de algoritmos extras que vocês utilizarem, \texttt{makefiles},
código de testes e um relatório no formato \texttt{pdf}.

\section{Tecnologias}

Esta seção descreve brevemente algumas tecnologias usadas no EP2.  O monitor
estará disponível na \textbf{Sala 120} para tirar dúvidas do EP2,
envie um e-mail para \texttt{pedro.bruel@gmail.com} para
marcar um horário.

\subsection{Acesso a GPUs da NVIDIA}

Vocês podem usar quaisquer GPUs da NVIDIA a que tenham acesso, mas também
podemos fornecer acesso às GPUs do nosso laboratório no CCSL para todos que
precisarem. Teremos que gerenciar os conflitos de uso, pois só um grupo poderá
usar uma dada GPU por vez. Cadastrem seus grupos usando o questionário no PACA
e forneçam um e-mail para contato.

\subsection{Shell scripting \& GNU \texttt{screen}}

Para deixar os experimentos rodando na sua máquina, faça o seguinte:

\begin{lstlisting}
$ screen
$ ./my_experiments.sh
<Ctrl+A><D>
\end{lstlisting}

O comando \texttt{screen} lança uma seção da qual você pode se desconectar sem
parar a execução de um comando. A sequência \texttt{<Ctrl+A>} seguida de
\texttt{<D>} desconectará você da sessão. Para voltar, basta executar:

\begin{lstlisting}
$ screen -r
\end{lstlisting}
%$

\subsection{\LaTeX}

Instalem o \LaTeX{} na máquina que vão usar para escrever o relatório e usem o
arquivo \texttt{enunciado\_ep2.tex} e o \texttt{Makefile} no repositório do EP2
como modelo.

\section{Critério de Avaliação}

A nota do EP2 vai de \textbf{0.0} a \textbf{10.0}, e a avaliação será feita da
maneira descrita a seguir, se os alunos concordarem.

\begin{itemize}
    \item Implementação: \textbf{6.0}
    \begin{itemize}
        \item Código compila sem erros e \textit{warnings}: \textbf{2.7}
        \item Código executa sem erros e produz o resultado correto: \textbf{2.7}
        \item Boas práticas de programação e clareza do código: \textbf{0.6}
    \end{itemize}
    \item Relatório: \textbf{4.0}
    \begin{itemize}
        \item Apresentação e Análise dos Experimentos: \textbf{3.4}
        \item Clareza do texto e figuras: \textbf{0.6}
    \end{itemize}
\end{itemize}

O monitor estará disponível na \textbf{Sala 120} para tirar dúvidas do EP2,
envie um e-mail para \texttt{pedro.bruel@gmail.com} para marcar um horário.
Bom EP!

%% References
%%
%% Following citation commands can be used in the body text:
%% Usage of \cite is as follows:
%%   \cite{key}          ==>>  [#]
%%   \cite[chap. 2]{key} ==>>  [#, chap. 2]
%%   \citet{key}         ==>>  Author [#]

%% References with bibTeX database:

%% \bibliographystyle{model1-num-names}
%% \bibliography{sample.bib}

%% Authors are advised to submit their bibtex database files. They are
%% requested to list a bibtex style file in the manuscript if they do
%% not want to use model1-num-names.bst.

%% References without bibTeX database:

% \begin{thebibliography}{00}

%% \bibitem must have the following form:
%%   \bibitem{key}...
%%

% \bibitem{}

% \end{thebibliography}


\end{document}

%%
%% End of file `elsarticle-template-1-num.tex'.
